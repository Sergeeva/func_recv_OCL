% -*- mode: LaTeX; mode: TeX-PDF; coding: utf-8  -*-

\section*{Эффективные параллельные 
  алгоритмы восстановления данных по известным значениям 
  в узлах равномерной сетки}
  %с применением 
  %аппарата средних В.~А.~Стеклова}

\label{sec:func_recv_annotate}

\bigskip

\textit{
  Рассматриваются параллельные алгоритмы 
  восстановления данных по известным значениям в узлах равномерной сетки.
%  методами %весового суммирования 
%  с применением аппарата средних В.~А.~Стеклова.
  Исследуются способы распараллеливания по данным и организации вычислений.
%  для решения поставленной задачи.
  Параллельные алгоритмы сформулированы в терминах
  %%модели массивно-синхронного параллелизма реального времени (МСПРВ)
  модели блочно-синхронно-конвейерного параллелизма (БСКП)
  для обработки потока данных в реальном времени. %(реального времени) 
  %модели вычислительной системы обработки информации
  %в реальном времени
  %и, таким образом подходят для целого класса вычислительных систем,
  %которые описывает такая модель.
  Произведена оценка вычислительной и коммуникационной сложности.
  %  и показана эффективность рассмотренного подхода к распараллеливанию.
  Выведены зависимости между размерностями задачи
  и параметрами модели БСКП. %вычислительной системы.
  Описана программная реализация предложенных алгоритмов
  в виде библиотеки функций. %на языке Си.
%  Приведены результаты работы программы на тестовых данных.
}



%%% Local Variables: 
%%% mode: latex
%%% TeX-master: "paper_func_recv"
%%% End: 



