% -*- mode: LaTeX; mode: TeX-PDF; coding:utf-8 -*-


\label{sec:func_recv_intro}

\emph{Практическое значение восстановления многомерных данных тут дб более широкое.}
Задачи восстановления данных возникают в разных областях обработки информации,
в том числе в задачах прикладной гидроакустики. 

% \emph{Например.}
% Задача восстановления данных 
% %функции нескольких переменных 
% по 
% %ее 
% известным значениям
% в узлах некоторой сетки имеет большое практическое %прикладное 
% значение. Ее часто приходится
% решать как самостоятельную задачу, кроме того, она является элементом решения
% многих вопросов прикладного характера. %, в частности, цифровой обработки
% %многомерных сигналов. 

При этом, зачастую, необходимо обработать достаточно большие объёмы данных.
Представляется актуальным поиск и реализация эффективных параллельных алгоритмов
для решения описанной задачи.

В настоящей работе мы остановимся на вычислительных аспектах
решения задачи восстановления данных в многомерном случае. 


В работе рассматривается параллельная реализация
алгоритма,
%предложенный 
В.~В.~Жука~\cite{book_Zhuk},
обладающего низкой вычислительной сложностью
и хорошим потенциалом в смысле параллелизма по данным. 

<? Надо бы отметить, что это обобщение нашей работы~\cite{my_paper_LETI_21}>


%На его основе  
%построены эффективные параллельные алгоритмы в терминах модели
%блочно-синхронно-конвейерного параллелизма (БСКП),
%которая объединяет в себе модель вычислительной системы
%и формальное описание обработки потока данных в реальном времени,
%представленные в работе~\cite{my_paper_LETI_MCS}.

В настоящей работе развивается
хорошо изученный в литературе
(например, в работах~\cite{paper_Val_BSP_90, paper_BSPStreaming,  paper_McColl_Tis})
подход к проектированию
алгоритмов с применением моделей параллельных вычислений.


С другой стороны, отметим работу~\cite{paper_Mas_recv},
%\emph{а там многомерный случай??} 
где рассматриваются вопросы параллельной реализации
похожих алгоритмов восстановления данных.



%\emph{Отображение массового параллелизма по данным на архитектуру GPU.}
%\emph{Обзор моделей применительно к GPU}



%%% Local Variables: 
%%% mode: latex
%%% TeX-master: "paper_func_recv"
%%% End: 



