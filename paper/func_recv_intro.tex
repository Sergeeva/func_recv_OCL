% -*- mode: LaTeX; mode: TeX-PDF; coding:utf-8 -*-


\label{sec:func_recv_intro}

%Изучению различных вопросов задачи восстановления посвящена огромная литература,
%авторами которой являются как математики, так и специалисты в других областях науки
%и техники. В частности, методы решения данной задачи даны в работах Д. Шепарда [1],
%Р. Харди [2], Дж. Дюшона [3]. Р. Франк в работе [4] дал обзор ряда подходов к решению этой
%задачи.
%1. Shepard D. A two-dimensional interpolation function for irregularly-spaced data // ACM’68:
%Proceedings of the 1968 23 rd ACM national conference. New York, 1968. P. 517–524.
%2. Hardy R.L. Multiquadric equations of topography and other irregular surfaces // Journal of
%geophysical research. 1971. Vol. 76. No 8. P. 1905–1915.
%3. Duchon J. Interpolation des fonctions de deux variables suivant le principe de la flexion des
%plaques minces // ESAIM: Mathematical Modelling and Numerical Analysis – Modélisation
%Mathématique et Analyse Numérique. 1976. Vol. 10. No 3. P. 5–12.
%4. Franke R. Scattered data interpolation: Tests of some methods // Mathematics of Computation .
%1982. Vol. 38. No 157. P. 181–200.


%– Восстановление поврежденных изображений [22–24].
%– Способы решения задачи восстановления функций могут быть использованы для
%восстановлении поврежденных изображений, например, если часть данных изображения
%утеряна, то недостающие могут быть восстановлены за счет имеющихся.
%– Построение поверхностей [25, 26] и автоматизированное геометрическое
%проектирование [27–32, 14].
%– Основная идея здесь состоит в том, чтобы обеспечить такое представление кривых
%и поверхностей, с которыми можно легко работать на компьютере, то есть легко хранить
%и отображать на экране ЭВМ.
%– Промышленное проектирование.
%Обычно проектировщик имеет описание кабины автомобиля, корпуса судна, флюзеляжа
%самолета, сложных деталей двигателей и т. д. в виде дискретного набора точек. Чтобы
%получить требуемый объект, нужно описать эти точки как лежащие на некоторой кривой или
%поверхности [33].

%22. Uhlir K., Skala V. Radial basis function use for the restoration of damaged images //
%M. Viergever, K. Wojciechowski, B. Smolka et al. (eds.). Computer Vision and Graphics. Vol. 32 of
%Computational Imaging and Vision. Dordrecht, 2006. P. 839–844.

%23. Perez P., Gangnet M., Blake A. Poisson image editing // ACM Transactions on Graphics. 2003.
%Vol. 22. No 3. P. 313–318.

%24. Lewis J.P. Lifting detail from darkness // SIGGRAPH’01: Proceedings of the SIGGRAPH 2001
%conference Sketches and Applications. ACM Press, 2001.

%25. Carr J.C., Beatson R.K., Cherrie J.B. et al.
%Reconstruction and representation of 3d objects with
%radial basis functions // SIGGRAPH’01: Proceedings of the 28 th annual conference on Computer
%graphics and interactive techniques. New York, 2001. P. 67–76.
%26. Lewis J.P. Lifting detail from darkness // SIGGRAPH’01: Proceedings of the SIGGRAPH 2001
%conference Sketches and Applications. ACM Press, 2001.

%27. Pighin F., Hecker J., Lischinski D.
%Synthesizing realistic facial expressions from photographs //
%SIGGRAPH’98: Proceedings of the 25 th annual conference on Computer graphics and interactive
%techniques. New York, 1998. P. 75–84.

%28. Noh J.-Y., Neumann U. Expression cloning // In SIGGRAPH’01: Proceedings of the 28 th annual
%conference on Computer graphics and interactive techniques. New York, 2001. P. 277–288.

%29. Joshi P., Tien W.C., Desbrun M., Pighin F. Learning controls for blend shape based realistic
%facial animation // SCA’03: Proceedings of the 2003 ACM SIGGRAPH / Eurographics symposium
%on Computer animation. Aire-la-Ville, 2003. P. 187–192.

%30. Lewis J.P., Cordner M., Fong N. Pose space deformation: A unified approach to shape
%interpolation and skeleton-driven deformation // SIGGRAPH’00: Proceedings of the 27 th annual
%conference on Computer graphics and interactive techniques. New York, 2000. P. 165–172.

%31. Sloan P.-P.J., Rose C.F., Cohen M.F. Shape by example // In SI3D’01: Proceedings of the 2001
%symposium on Interactive 3D graphics. New York, 2001. P. 135–143.

%32. Kurihara T., Miyata N. Modeling deformable human hands from medical images // Proceedings
%of the 2004 ACM SIGGRAPH Symposium on Computer Animation (SCA-04). 2004. P. 357–366.

%33. Квасов Б.И. Методы изогеометрической аппроксимации сплайнами. М.; Ижевск,
%2006. – 413 с.


%Задача восстановления данных по известным значениям в узлах некоторой сетки имеет большое практическое значение.
%Она имеет множество приложений таких как моделирование и построение различных геометрических объектов,
%численное решение уравнений математической физики и многие другие в разных областях математики,
%геологии, биологии, обработки сигналов и т. п.
%Рассмотрим подробнее некоторые из них.
%Восстановление поврежденных изображений.
%Способы решения задачи восстановления функций могут быть использованы для восстановлении поврежденных изображений,
%например, если часть данных изображения утеряна, то недостающие данные могут быть восстановлены за счет имеющихся.
%Построение поверхностей и автоматизированное геометрическое проектирование.
%Основная идея здесь состоит в том, чтобы обеспечить такое представление кривых и поверхностей,
%с которыми можно легко работать на компьютере, т.е. легко хранить и отображать на экране ЭВМ.  
%Промышленное проектирование.
%Обычно проектировщик имеет описание кабины автомобиля, корпуса судна, флюзеляжа самолета,
%сложных деталей двигателей и т.д. в виде дискретного набора точек.
%Чтобы получить требуемый объект, нужно описать эти точки как лежащие на некоторой кривой или поверхности [9].


%Задачи восстановления данных возникают в разных областях обработки информации,
%ей посвящена обширная литература.

Восстановление
% многомерных
%((многомерных --- двусмысленно, кстати забыли оговорить обл заначений наших функций,
%теперь лучше оставить как есть))
данных является составной частью решения различных 
% вычислительнотрудоёмких
задач в таких областях как, компьютерная графика,
промышленное проектирование,
обработка сигналов и многих других
(см., например,~\cite{book_Kvasov, paper_recv_1, paper_recv_3}).
%где необходимо обработать достаточно большие объёмы данных.
%((При решении таких задач зачастую применяются параллельные <высокопроизводительные>
%(многопроцессорные или многоядерные) вычислительные системы.
%--- думаю убрать т к норм не сформулировать)) 
Представляется актуальным поиск и реализация эффективных параллельных алгоритмов
для решения задачи восстановления данных.

%В настоящей работе мы %остановимся на
%рассматриваем
Работа посвящена 
вычислительным аспектам 
решения задачи восстановления данных в многомерном случае. 
% \emph{Например.}
% Задача восстановления данных 
% %функции нескольких переменных 
% по 
% %ее 
% известным значениям
% в узлах некоторой сетки имеет большое практическое %прикладное 
% значение. Ее часто приходится
% решать как самостоятельную задачу, кроме того, она является элементом решения
% многих вопросов прикладного характера. %, в частности, цифровой обработки
% %многомерных сигналов. 
Рассматривается, также как и в предыдущей работе авторов~\cite{my_paper_LETI_21},
параллельная реализация
алгоритма
В.~В.~Жука~\cite{book_Zhuk}. %,
%((обладающего
%низкой вычислительной сложностью
%хорошим потенциалом в смысле параллелизма по данным. --- было, нельзя так повторяться)) 
Отметим также работу~\cite{paper_Mas_recv},
% С другой стороны, отметим работу~\cite{paper_Mas_recv},
%\emph{а там многомерный случай??} 
где рассматриваются вопросы параллельной реализации
похожих алгоритмов восстановления данных.


%На его основе  
%построены эффективные параллельные алгоритмы в терминах модели
%блочно-синхронно-конвейерного параллелизма (БСКП),
%которая объединяет в себе модель вычислительной системы
%и формальное описание обработки потока данных в реальном времени,
%представленные в работе~\cite{my_paper_LETI_MCS}.

% В настоящей работе развивается
% хорошо
%Применяется изученный в литературе
%(см., например, \cite{paper_Val_BSP_90, paper_BSPStreaming,  paper_McColl_Tis})
%подход к проектированию
%алгоритмов с применением моделей параллельных вычислений.

При проектировании алгоритмов 
важно выявить те части,
которые могут выполняться параллельно и то как будут использоваться 
имеющиеся в распоряжении вычислительные ресурсы.
Широко изучающемся в литературе
(см., например, \cite{paper_Val_BSP_90, paper_BSPStreaming, paper_Val_multiBSP_08})
инструментом для решения поставленных вопросов являются
модели параллельных вычислений, которые отражают ключевые особенности
некоторого класса целевых архитектур.
%Современные модели паралелльных вычислений обычно специализированные,
%т.е. строятся для некоторого класса
%алгоритмов и некоторого класса целевых архитектур.
С другой стороны, возможность эффективной реализации на том или ином классе
целевых архитектур зависит от характеристик алгоритма.
Поскольку для рассматриваемого алгоритма 
характерен параллелизм по данным, 
%Исходя из характеристик рассматриваемого алгоритма
выбрана модель массового параллелизма,
описываемая стандартом Open CL
%((Open CL --- надо с пробелом! --- исправить где без))
(Open Computational Langua\-ge)~\cite{doc_OCL}.



%\emph{Отображение массового параллелизма по данным на архитектуру GPU.}
%\emph{Обзор моделей применительно к GPU}



%%% Local Variables: 
%%% mode: latex
%%% TeX-master: "paper_func_recv"
%%% End: 



