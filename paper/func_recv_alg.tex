% -*- mode: LaTeX; mode: TeX-PDF; coding: utf-8  -*-

\label{sec:func_recv}

\subsection*{Алгоритм восстановления данных}

Восстановление значения в произвольном  узле $x$    мелкой сетки %в узле $x$,
$r$ ($r\in\mathbb{Z}_+^n$) раз непрерывно дифференцируемой функции $\varphi$
выполняется следующим образом: 
\begin{gather}
  \label{eq:recv_common}
  g_{h,r+\mathbf{2}^n}(\varphi, x) = \sum_{k\in  \Delta_{K,r}}
   \varphi(u^{(k)})
   \psi_{h, r+\mathbf{2}^n}(x-u^{(k)}),%\\ \notag \text{где}\ 
 \end{gather}
 где $ \Delta_{K,r}=\left[-\lfloor{(r+\mathbf{1}^n)/2}\rfloor:
   K-\mathbf{1}^n+\lfloor{(r+\mathbf{1}^n)/2}\rfloor\right]$.  %\cap\mathbb{Z}_+^n

 
Из формулы~\eqref{eq:recv_common} следует, что для вычислений
могут быть нужны дополнительные
узлы по краям исходной сетки.
Способы выбора значений в этих узлах зависят
от специфики задачи для которой применяется формула~\eqref{eq:recv_common}  
и в данной работе не рассматриваются.





%%% Local Variables: 
%%% mode: latex
%%% TeX-master: "paper_func_recv"
%%% End: 



