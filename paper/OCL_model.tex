% -*- mode: LaTeX; mode: TeX-PDF; coding: utf-8  -*-


\label{sec:OCLprog}


%\subsection*{Основные обозначения}
\subsection*{Описание программы восстановления данных на OCL}
%\emph{OCL реализация алгоритма восстановления данных}
%Выполнена реализация предложенной вычислительной схемы для
%двухмерного случая в стандарте Open CL (Open Computational Language)~\cite{doc_OCL}.

Выбрана модель массового параллелизма,
описываемая стандартом Open CL (Open Computational Language)~\cite{doc_OCL}.

Open CL является открытым и широко применяемым стандартом в области вычислений
общего назначения на((в?)) гетерогенных (неоднородных) системах
(см., например,~\cite{paper_OCL_Komdiv}). %[ссылку или пояснить?].

Одним из известных примеров гетерогенных систем являются системы,
содержащие в своём составе графические процессоры (GPU --- Graphic Processor Unit).
%Современные графические процессоры (GPU --- Graphic Processor Unit)
%включают в себя большое количество простых 
%вычислительных элементов((потом термин!)). %поэтому 
%В силу специфики архитектуры
Такие системы в силу архитектурной специфики  
хорошо подходят для эффективной реализации задач,
обладающих массовым параллелизмом.
Вычислительные системы с GPU в настоящее время находят применение не только для
графического рендеринга, но и для ускорения ((выполнения)) различного рода
вычислений общего назначения. 
%(известной как GPGPU --- General Purpose GPU).
%В силу специфики архитектуры такие системы 
%хорошо подходят для эффективной реализации задач,
%обладающих массовым параллелизмом.


%% На уровне гетерогенной системы: управляющий процессор -- ускоритель
Модель гетерогенной вычислительной системы в стандарте Open CL 
состоит из управляющего процессора (Host)
и вычислителей (Compute Device),
содержащих, в свою очередь, 
модули (Compute Unit),
которые включают в себя вычислительные элементы (Processing Element).
<или work-item --- рабочий элемент>
Представление вычислительной платформы в Open CL
показано на рисунке~\ref{fig:OCL_platform}.

<возможно, будет один обобщённый рисунок>
\begin{figure}[h!]
  \centering
  \includegraphics[width=\textwidth,height=7cm]{OCLmodel_platform} 
  \caption{Состав вычислительной платформы в Open CL}
  \label{fig:OCL_platform}
\end{figure}
\FloatBarrier

\begin{figure}[h!]
  \centering
  \includegraphics[width=\textwidth,height=7cm]{OCLmodel_mem} 
  \caption{Иерархия памяти вычислительной платформы в Open CL}
  \label{fig:OCL_wg}
\end{figure}
\FloatBarrier




%% Модель памяти
%Иерархия памяти вычислительных ускорителей представляется следующим образом
%Вычислительные устройства
Вычислители
имеют следующую иерархию памяти 
(см. рисунок~\ref{fig:OCLmodel_mem}).
Всем вычислительным элементам доступна глобальная память. 
%((VRAM RAM неудачные чисто условные сокр-я)).
Через глобальную память также происходит обмен данными с памятью управляющего процессора. 
Каждый модуль 
обладает локальной памятью, к которой имеют доступ
его вычислительные элементы.
Вычислительные элементы имеют внутреннюю  память (регистры), доступную только им.

%Execution of an OpenCL program occurs in two parts: kernels that execute on one or more
%OpenCL devices and a host program that executes on the host. The host program defines the
%context for the kernels and manages their execution.
Исполняемая программа, разработанная в стандарте OpenCL,
содержит:  
вычислительные ядра (kernels), выполняющиеся
параллельно 
в SIMD режиме 
на вычислительных элементах ускорителей
и управляющую программу, которая
определяет контекст и 
управляет запуском вычислительных ядер.

!<Или тут может быть ещё более общий рисунок!>
((вместо текста лучше: обобщенная схема разработанногов модел OCL ПО приведена на рисунке))

%% Параллелизм на уровне вычислительных элементов
%Модель параллелизма %программирования
%в OpenCL
%на уровне GPU выражается как SIMT (Single Instruction Multiple Threads) --- терминология CUDA.

%Множество потоков в рамках которых выполняется
%код вычислительных ядер (kernel) выполняется %параллельно
%на множестве вычислительных элементов.

На уровне абстрации модели вычислений в Open CL ((может как нибудь проще?))
вычислительные элементы организованы в рабочее пространство: ((---))
$n$-мерную решётку,
размером $NDRange$.
Помимо этого вся решётка разбивается на $n_{wg}$-мерные рабочие группы
размером $NDWorkGroup$. 
Для каждого вычислительного элемента в такой модели определены:  
$GlobalID$ %(в общем случае вектор)
--- индекс вычислительного элемента из $[0, NDRange - \mathbf{1}^n]$;
%в пространстве размерности задачи (NDRange).
$LocalID$ --- индекс вычислительного элемента в рабочей группе
из $[0, NDWorkGroup - \mathbf{1}^n]$. 
%пространстве размерности рабочей группы (Work Group).
На рисунке~\ref{fig:OCL_wg} приведён пример
часто встречающегося на практике двумерного рабочего пространства.

\begin{figure}[h!]
  \centering
  \includegraphics[width=\textwidth,height=7cm]{OCLmodel_wg} 
  \caption{Пример модели рабочего пространства Open CL}
  \label{fig:OCL_wg}
\end{figure}
\FloatBarrier


%%% Local Variables: 
%%% mode: latex
%%% TeX-master: "paper_func_recv"
%%% End: 



